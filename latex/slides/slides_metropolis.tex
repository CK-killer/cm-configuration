\documentclass[utf8,10pt]{beamer}
% Packages
\usepackage{amsmath, amsthm, amssymb, amsfonts, hyperref, mathrsfs}%美国数学学会的包+?
\usepackage{geometry} %控制界面
\usepackage{bookmark}
\usepackage{fancyhdr} % header & footer
\usepackage{appendix} % 附录
\usepackage{tikz} %作图
\usepackage{graphicx} %插入图片的宏包
\usepackage{float} %设置图片浮动位置的宏包
\usepackage{subfigure} %插入多图时用子图显示的宏包
\usepackage{listings} %引用代码
\usepackage{physics,mathtools} %物理数学工具
\usepackage{verbatim} %多行注释工具
\usepackage{framed}
\usepackage{multicol} %分列
\usepackage{comment}
\usepackage{braket} %Dirac braket
\usepackage{url}
\usepackage{makecell}
\usepackage{bm}
\usepackage{appendixnumberbeamer} %不编号附录页
\usepackage{pgfplots}
\usepackage{xcolor,svg}
\usepackage{beamercolorthemeowl}
\usepackage{hyperref} %可以引用网页
\hypersetup{
    colorlinks=true,
    linkcolor=cyan,
    filecolor=blue,      
    urlcolor=blue,
    citecolor=cyan,
} %设置引用链接的颜色
\usepackage{xeCJK}

\usepackage{booktabs} %表格
\usepackage[scale=2]{ccicons} %Creative Commons
\usepackage{pgfplots} %作图
\usepgfplotslibrary{dateplot}
\usepackage{xspace} %自动判断是否需要在宏定义后面加空格
\newcommand{\themename}{\textbf{\textsc{metropolis}}\xspace}
%\usepackage[fontset=founder]{ctex}% 自由切换字体宏包
% Beamer theme
\usetheme[block=fill, progressbar=frametitle]{metropolis}%sectionpage=none,
\useoutertheme{infolines}
\useinnertheme{metropolis}
%\usecolortheme{custom}
\setbeamertemplate{blocks}[rounded][shadow=true]
\setbeamertemplate{items}[ball]
\setbeamertemplate{sections/subsections in toc}[ball]
% Windows font
% 中文默认字体
\setCJKmainfont{Noto Serif CJK SC}[BoldFont={Noto Serif CJK SC Bold}, ItalicFont=]
%
\bibliographystyle{plain} % 引用样式
\setbeamertemplate{bibliography item}[text] % 参考文献样式
\everymath{\displaystyle} % display

\title{A minimal example}
\date{\today}
\author{Jinming He}
\institute{Physics Department, USTC}
\titlegraphic{\hfill\includegraphics[height=1.5cm]{logo_ustc.pdf}}

\begin{document}
  \maketitle

  \begin{frame}{Table of contents}
    \setbeamertemplate{section in toc}[sections numbered]
    \tableofcontents%[hideallsubsections]
  \end{frame}

  \section[Intro]{First Section}

  \subsection{First Subsection}
  \begin{frame}{First Frame}
    Hello, world!你好\footnote{Hola},我是USTCer\cite{gqm}\cite{qm2}
  \end{frame}

  \begin{frame}{Animation}
  \begin{itemize}[<+- | alert@+>]
    \item \alert<4>{This is\only<4>{ really} important}
    \item Now this
    \item And now this
  \end{itemize}
  \end{frame}

  \section[second section]{Second One}


  \begin{frame}{Second Frame}
    \begin{block}{Potential Problems}
		  This title format\footnote{this is a footnote} is not as problematic as the \texttt{allsmallcaps} format, but basically suffers from the same deficiencies. \alert{So please have a look at the documentation if you want to use it.}
	  \end{block}

    \begin{columns}[T,onlytextwidth]
    \column{0.47\textwidth}
	  \begin{itemize}
      \item Regular
      \item \textit{Italic}
      \item \textsc{Small Caps}
      \item \textbf{Bold}
    \end{itemize}
    \column{0.47\textwidth}

    This is a text in second column.

    \end{columns}
  \end{frame}

\begin{frame}{References}
  \bibliography{citation}
\end{frame}


\end{document}